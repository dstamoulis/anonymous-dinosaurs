%% ECSE 420 -- Project proposal

%
\documentclass[conference]{IEEEtran}
%
% If IEEEtran.cls has not been installed into the LaTeX system files,
% manually specify the path to it like:
% \documentclass[conference]{../sty/IEEEtran}


%My packages
\usepackage{graphicx,dblfloatfix}
\graphicspath{{./figures/}}
\usepackage[xetex,colorlinks=true,urlcolor = black, 
linkcolor=black,citecolor=black]{hyperref}


% correct bad hyphenation here
\hyphenation{op-tical net-works semi-conduc-tor}


\begin{document}

% paper title
% can use linebreaks \\ within to get better formatting as desired
\title{ECSE 548: Modified 8-bit Booth Multiplier}
%\author{\IEEEauthorblockN{Marco Kassis\IEEEauthorrefmark{1},
%Aryan Mojtahedi\IEEEauthorrefmark{1},
%Dimitrios Stamoulis\IEEEauthorrefmark{1} and
%Louis-Charles Trudeau\IEEEauthorrefmark{1}\IEEEauthorrefmark{2}}
%\IEEEauthorblockA{\IEEEauthorrefmark{1}McGill University, Montreal, Canada ~~~~~ \IEEEauthorrefmark{2}Ecole de technologie superieure, Montreal, Canada}
\author{\IEEEauthorblockN{Marco Kassis,
Aryan Mojtahedi,
Dimitrios Stamoulis and
Louis-Charles Trudeau}
\IEEEauthorblockA{Department of Electrical and Computer Engineering, McGill University, Montreal, Canada}
}

% make the title area
\maketitle
\IEEEpeerreviewmaketitle



\begin{abstract}
%\boldmath
A Multiplier is one of the most important elements of any modern processor, whether it be for the sake of pure multiplication to evaluate for arithmetic purposes or for other functions that make use of that function. It is one of the processes that consume a significant amount of processing time, since it is a series of additions, making it very important to optimize. This report presents the modified booth algorithm applied on an 8-bit level. The multiplier’s performance is then compared to that of the Baugh-Wooley Multiplier with respect to area, transistor count and delay. Integration within the MIPS core is also implemented and presented in this paper. The CAD tool used for the design process is Electric, while testbenching was done by Verilog file exports on ModelSim, while the test vectors were generated using a C program.
\end{abstract}


% Note that keywords are not normally used for peerreview papers.

\begin{IEEEkeywords}
Multiplier; Modified Booth Algorithm; MIPS; Baugh-Wooley Multiplier.
\end{IEEEkeywords}



\section{Introduction}
\IEEEPARstart{T}{he} 
multiplication of 2 given 8-bit vectors, M and N, results in a vector larger than both of them, namely 16-bits. To regularly perform this multiplication, it has to be done bitwise, generating N partial products of the same size as M, shifted and then added to compute the value of the multiple.

Accordingly, using the most basic form of the array multiplier, we realize that that the most significant bit (MSB) takes 2M-2 time units till it reaches its final output. This is a significant waste of processing time, given that the multiplier is essentially on module within the entire processor. Hence, a logical optimization is to group the partial products together to reduce their number, hence reducing the number of additions that need to be done, which would reduce the amount of time taken to compute the product.

Modified booth algorithm introduces this type of grouping based on the inputs being inserted into the multiplier, which is discussed in the upcoming section.


\subsection{Aim}

To produce a fast 8-bit multiplier, reducing both area and time delay compared to the regular Baugh-Wooley Multiplier. The multiplier implements the ModifiedBooth Logic and is integrated within the reduced 8-bit MIPS processor.


\section{Booth Logic}

The booth logic depends on an important metric r, which defines the number of partial products to be grouped together. Accordingly, the radix of the system is defined to be 2r. The radix itself defines the weight by which each and every of the partial products differ from the one to the following. We hence observe that the weight of the r=2, is a radix of 4, since grouping together 2 elements will result in a shift of 2 between each of the partial products which is equivalent to multiplying by 4 from one to the following. The utilization of a radix-4 system provides a reduction in the number of partial products reaching $\frac{M+1}{2}$
instead of M, which is almost half, hence a significant improvement. The size of each vector is N+1.
a
The radix-4 system is the one used in this project. The 2 vectors being multiplied will be referred to as X and Y for convenience within this section. This leads to having the 4 possible values of the partial products of Y being 0, Y, 2Y and 3Y. 2Y can be easily achieved using the single shift of the partial product; however, 3Y would require an adder, eliminating the whole point of the booth logic. Accordingly, a workaround would be to consider 3Y as –Y+4Y, while 2Y would become -2Y+4Y, with the first part of the sum accounted for earlier then the second part following that. This leads to observing of 3 bits at a time, the MSB from the previous subvector taken, with the other current 2. Accordingly FIGURE1 shows the truth table for this type of implementation.

The single, double and negative signals are the selected values that signify the operation to be implemented, which is elaborated on in the encoder part of next section. The dot diagram with sign extension of the booth logic is seen in FIGURE2.







% that's all folks
\end{document}

